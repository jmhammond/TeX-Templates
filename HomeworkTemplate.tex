\documentclass[11pt]{article}


% ========================================================================
% preamble 
% ========================================================================

% margins, size, formatting
\oddsidemargin=.25in
\evensidemargin=.25in
\topmargin=-.5in
\textwidth=6in
\textheight=9in
\pagestyle{empty}

\def\Z{{\mathbb{Z}}}  % shortcuts
\def\R{{\mathbb{R}}}
\def\C{{\mathbb{C}}}
\def\Im{\mbox{Im }}
\def\Re{\mbox{Re }}
\def\ds {\displaystyle}
\def\and {\mbox{ and }}

\newcommand{\pd}[2]{\frac{\partial #1}{\partial #2}}
\newcommand{\pdd}[2]{\frac{\partial^2 #1}{\partial #2 ^2}}
\newcommand{\pddd}[2]{\frac{\partial^3 #1}{\partial #2 ^3}}
\newcommand{\powerseries}{\sum_{n=0}^\infty a_n (z-z_0)^n}
\newcommand{\Powerseries}[2]{\sum_{n=0}^\infty #1 (#2)^n}

% packages for fancy fonts, symbols, thm/proof environments, etc
\usepackage{amsmath,amssymb,amsthm}


\newtheorem{theorem}{Theorem}

% laura's special format for HW problems
\newsavebox{\probinput}
\newenvironment{problem}[1]%
  {\vspace{\baselineskip}%
   \ifnum #1>9 \else \hspace{.78ex} \fi {\large \bf#1.$\;$}%
   \begin{lrbox}{\probinput}%
   \begin{minipage}[t]{.9175\textwidth}}%
  {\end{minipage}%
   \end{lrbox}%
   \usebox{\probinput}}
\newsavebox{\subinput}
\newenvironment{subprob}[1]%
  {\vspace{.5\baselineskip}%
   {\bf #1)$\;$}%
   \begin{lrbox}{\subinput}%
   \begin{minipage}[t]{.965\textwidth}}%
  {\end{minipage}%
   \end{lrbox}%
   \usebox{\subinput}}


% ========================================================================
% body 
% ========================================================================

\begin{document}

% header =================================================================
{\large \bf 7940 Homework 1.1-1.6 \hfill John Hammond}
\vspace{\baselineskip}

% problem ================================================================
\begin{problem}{1} 
Prove something about $f : \Omega \to \C$ being analytic.

\emph{Proof.}  This is my proof.   $\ds \powerseries$

It has something to do with partial derivatives $\pd{f}{x}$

\qed
\end {problem}

\begin{problem}{15}
Let $a, b \in \C$ with $a \not=0$, and let $T(z) = az + b, z \in \C$

\begin{subprob}{(i}
Show that $T$ maps the circle $C(z_0, r) $ onto the circle $C(T(z_0), r|a|)$.
\end{subprob}
\begin{subprob}{(ii}
For which choices of $a$ and $b$ will $T$ map $C(0,1)$ onto $C(1 + i, 2)$?
\end{subprob}
\begin{subprob}{(iii}
In (ii), is it possible to choose $a$ and $b$ so that $T(1) = -1 + 3i$?
\end{subprob}
\emph{Proof.}

\qed
\end{problem}






\end{document} 